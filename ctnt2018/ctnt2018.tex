\documentclass{beamer}
%\usepackage{minted}
\newtheorem{proposition}{Proposition}
\newtheorem{algorithm}{Algorithm}
\hypersetup{colorlinks,linkcolor=,urlcolor=blue}

%\usepackage[colorlinks=true]{hyperref}
\begin{document}
\begin{frame}
	\begin{center}
		Lenstra's Elliptic Curve Factoring Method \\
		Connecticut Number Theory Summer School \\
		May, 2018 \\
		\bigskip
		Jeremy Teitelbaum
	\end{center}
\end{frame}
\begin{frame}{The problem at hand}
	\begin{problem} Given a positive composite integer $N$, find a proper prime divisor of $N$.
	\end{problem}
\end{frame}
%\begin{frame}{Factoring is important}
%
%	{\small
%	Problema, numeros primos a compositis dignoscendi, 
%	hosque in factores suos primos resolvendi, ad gravissima ac utilissima totius arithmeticae pertinere, 
%	...Praetereaque scientiae dignitas requirere videtur, ut omnia subsidia ad solutionem 
%	problematis tam elegantis ac celebris sedulo excolantur.
%	
%	\medskip\noindent
%	The problem of distinguishing prime numbers from composite numbers and of
%	resolving the latter into their prime factors is known to be one of the most
%	important and useful in arithmetic. ..Further, the dignity of the
%	science itself seems to require that every possible means be explored for
%	the solution of a problem so elegant and so celebrated. 
%	
%\medskip\noindent
%	{\it Gauss,  Disquisitiones Arithmeticae (1801): Article 329}
%	}		
%			
%\end{frame}
%\begin{frame}{Cryptography}
%	\begin{block}{}
%	Gauss thought factoring was important and he was unaware of the role it
%	plays in the security of widely used public-key cryptographic systems.
%	\end{block}
%	\begin{block}{}
%		Although a major reason for current work on the problem, we won't get into the cryptographic applications in this talk.
%	\end{block}
%\end{frame}
	
\begin{frame}{Trial division is impractical}
	
\begin{block}{}
	The 'grade school' method to solve the factoring problem by systematically trying integers less than $N$ (or prime numbers less than $N$)
	and checking to see if you find a factor requires, in the worst case, on the order of $\sqrt{N}$ divisions.  
\end{block}
\begin{block}{}
	If a division takes, say, $10^{-12}$ seconds on some miracle computer, then factoring a $100$ digit number would require
	$10^{38}$ seconds or more than $10^{30}$ years.  (The universe is about $10^{10}$ years old.)
\end{block}
\begin{block}{}
	A different approach is needed.
\end{block}
\end{frame}

\begin{frame}{Overview of factoring methods}
	Modern methods of factoring fall into two categories:
	\begin{itemize}
		\item Methods based on algebraic groups, such as the $p-1$ method, the elliptic curve method, and generalizations.
		\item Sieve methods, such as the quadratic and number field sieves.
	\end{itemize}
\end{frame}
\begin{frame}{Overview of factoring}

	Typically, the algebraic group methods are used first to identify ``small
	factors'' of large numbers $N$; and once those are found, or ruled out, the
	sieve methods are used.

\bigskip\noindent In the best case these algorithms are believed to be
	sub-exponential, meaning that their running times grow more slowly than
	exponential in the number of digits of $N$; but they are far from polynomial
	time.

\bigskip\noindent
	The complexity of factoring is not known.  
	
\bigskip\noindent 
	There is a polynomial time algorithm for a ``quantum computer.''
\end{frame}
\begin{frame}{First make sure your number is composite}
	\begin{theorem}[Fermat] Suppose that $N$ and $a$ are integers with $(a,N)=1$.  If 
		$$a^{N-1}\not\equiv 1\pmod{N}.$$
	then $N$ is composite.
	\end{theorem}

	Fermat's theorem allows for a quick test of compositeness.  
	
	\begin{block}{The Fermat Test}
	Given $N$ (large), pick a random small $a$ and compute $a^{N-1}\bmod{N}$.
	If the result isn't $1$, $N$ is composite. 
	\end{block}

	\begin{definition}
	If $a^{N-1}\equiv 1\pmod{N}$, then $N$ is called a pseudoprime to base $a$.
	\end{definition}
\end{frame}
\begin{frame}{Pseudoprimes are rare}
	There are $21853$ pseudoprimes to base $2$ less than $25\times 10^{9}$.

	\bigskip\noindent
	If a number passes the Fermat test for a bunch of random bases, then spend your time trying to prove it prime
	rather than trying to factor it.

	\bigskip\noindent
	There are refinements to the Fermat test that are even more effective.

\end{frame}
\begin{frame}[fragile]{Efficient Modular Exponentiation}
	Applying the Fermat Test requires computing $a^x\bmod{N}$ where $x$ is
	large; and similar calculations are needed in the ECM method as well.

	\begin{proposition} $a^x\bmod{N}$ can be computed in time $O(\log x)$ for fixed $N$ and $a$. 
	\end{proposition}
	\begin{algorithm}
	\begin{verbatim}
	Set m=1 and s=a.
	While x>0:
	   if x is odd, set m=(m*s mod N)
	   set s=s*s
	   set x=x/2, rounding off
	return m as your answer
	\end{verbatim}
\end{algorithm}
\end{frame}
\begin{frame}{The $p-1$ algorithm}
	Suppose $N$ is (odd and) composite.   Then the multiplicative group of units $(\mathbf{Z}/N\mathbf{Z})^*$ is not cyclic, so it is a product of cyclic 
	groups by the fundamental theorem of abelian groups. 

	\bigskip\noindent
	The strategy of the $p-1$ method is to
	\begin{enumerate}
		\item pick a base $a$ (like $2$);
		\item try to find an exponent $M$ so that $a^{M}\equiv 1$ in one of the cyclic factors of $(\mathbf{Z}/N\mathbf{Z})^*$ but not all of them. 
		\item then $(a^{M}-1,N)$ will be a non-trivial factor of $N$.
	\end{enumerate}	

	\bigskip\noindent
	If $p$ is an odd prime factor of $N$, then we can try $M=K(p-1)$ for various $K$.  But how to find this $M$ if we don't know $p$?

\end{frame}
\begin{frame}{Smoothness}
	\begin{definition}
	An integer $N$ is called $B$-smooth if all the prime factors of $N$ are at most $B$.  It is called $B$-powersmooth if all the prime
	powers dividing $N$ are at most $B$.  
	\end{definition}

	For example, the number
	$$
	N=33452526613163807108170062053440751665152000000000
	$$
	is $41$-smooth. (It is $41!$).  It is divisible by $2^{164}$ and all the other prime powers dividing $41!$ so $41!$ is
	$2^{164}+1$ powersmooth.

	
\end{frame}
\begin{frame}{The $p-1$ method, 2}
	
	The hope for the $p-1$ method is that if $p$ is one  of the prime divisiors of our integer $N$ then $p$ has the property that $p-1$ is $B$-powersmooth
	for some not too big $B$. 

	\bigskip\noindent
	Then we take a integer $M$ that is divisible by powers of the primes less than $B$ hoping to get a multiple of $p-1$. 
	For example, take:
	$$
	M=\prod_{p\le B} p^{[\log_{p}(B)]}.
	$$

	\bigskip\noindent
	Then compute $(a^{M}-1,N)$ and see what happens.  If you don't find anything, make $B$ bigger.

\end{frame}
\begin{frame}{A simple example}
	Suppose $N=F_{5}=2^{2^5}+1$ is the fifth Fermat number.  Take $B=150$.  We can't use $a=2$ because clearly high powers of $a$ are going to be 
	$-1$ mod $N$; so let's try $a=3$. That doesn't work -- but $a=5$ does. Take $M=(128)*(81)*(25)(49)(121)(13)(17)\cdots(97)\cdots(149)$.  
	$$
	5^M-1 \equiv 1741227785\pmod{F_{5}}
	$$
	and $(1741227784,F_{5})=641$. 
\end{frame}

\begin{frame}{The Elliptic Curve Method}
	For the $p-1$ method to work, we have to be lucky enough to have a prime factor that is $B$-powersmooth for a relatively small $B$.

        \bigskip\noindent
	If the number $N$ we are trying to factor doesn't have this property, then the $p-1$ method won't work.


        \bigskip\noindent
	The elliptic curve method opens the door to more situations in which we can apply the idea of the $p-1$ method.
\end{frame}
\begin{frame}{ECM, cont'd}
	Suppose $N=UV$ where $U$ and $V$ are proper factors.  Let $E$ be an elliptic curve over $\mathbf{Z}$.  Then 
	$$
	E(\mathbf{Z}/N\mathbf{Z})=E(\mathbf{Z}/U\mathbf{Z})\times E(\mathbf{Z}/V\mathbf{Z}).
	$$
	Suppose that we can find a point $P$ on this curve mod $N$ so that a multiple $K$ of $P$ is zero in the first factor but not the second.
	
	\bigskip\noindent
	If we were to write $E$ in Weierstrass form, and the point $P$ in (reduced) homogeneous coordinates $[x(P):y(P):z(P)]$, then this condition
	would mean that $z(KP)$ is divisible by $U$ but not by $V$.  
	
	\bigskip\noindent
	In other words, $(z(KP),N)$ would give us a proper factor of $N$.
\end{frame}
\begin{frame}{ECM,3}

	If we were fortunate enough that (say) the order of the  first of the two factor groups
	$n=|E(\mathbf{Z}/U\mathbf{Z})|$ were $B$-powersmooth for a (relatively) small $B$,
	Then we could use the trick of the $p-1$ method and choose our $K$ to hopefully be divisible by $n$.

        \bigskip\noindent
	The Riemann hypothesis for elliptic curves over finite fields tells us that if $U$ is prime then $n$ is roughly $p$.

        \bigskip\noindent
        It seems reasonable to assume that these group orders are essentially random numbers of size roughly $p$.
	So the chance that $n$ is $B$-powersmooth is the same order as $p-1$ having that property.

        \bigskip\noindent
	But there are many elliptic curves! 
\end{frame}
\begin{frame}{ECM: An example}
Consider the $7^{th}$ Fermat number 
$$
F_{7}=2^{128}+1=340282366920938463463374607431768211457.
$$
A Fermat test to base $3$ confirms that $N$ is composite:
$$
3^{N-1}\equiv 47511664169441434718291075092691853899\not\equiv 1\pmod{N}
$$
\end{frame}
\begin{frame}{ECM: An example, cont'd}
Try many elliptic curves $E$ and a large but not hopelessly large $B$ so that, if $U$ is a factor of $N$, then
$|E(\mathbf{Z}/U\mathbf{Z})|$ is $B$-powersmooth.

\bigskip\noindent
Consider the family of elliptic curves $E_{a}:y^2=x^3+ax+1$ which has the obvious point $P=(0,1)$ on it.  

\bigskip\noindent
Consider $a$ as integers (for example in the range $[-100,100]$) and try $B=10000$.  Let
$$
B_{4}=\prod_{p<10000} p^{[\log (10000)/log(p)]}
$$
For each $a$, compute
$[B_4](P)$ in $E_{a}(\mathbf{Z}/N\mathbf{Z})$.  Use the same idea as in modular exponentiation to compute $[B_4](P)$ (repeated doubling of the point).  

\bigskip
In computing this, you have to do modular inversion mod $N$.  If that fails: you've found a factor of $N$.
\end{frame}
\begin{frame}{ECM: an example, cont'd}
A very naive search finds that 
$$
E_{-91}: y^2=x^3-91x+1
$$  
gives us our factorization and finds the factor 
$$N=UV=(59649589127497217)(5704689200685129054721)$$

\bigskip\noindent
Some quick pseudoprime tests suggests that this is prime (it is) and $U-1$ has a prime divisor with 15 digits and $V-1$
has a prime factor with $12$ digits,  so this factorization is unlikely to have been found by the $p-1$ method.
\end{frame}
\begin{frame}{Lenstra's ECM for theoretical purposes}
Let $N$ be an integer that is not a prime power and is relatively prime to
$6$. 
\begin{enumerate} 

\item Choose integers $v$ and $w$ and set
$$
k=\prod_{2}^{w} r^{e(r)}
$$
where $e(r)$ is maximal so that 
$r^{e(r)}\le v+2\sqrt{v}+1.
$
\item Draw three elements $a,x_0,y_0$ at random modulo $N$.  Let $b=y_0^2-x_0^3-ax$.  Then $P=(x_0,y_0)$ is a point on the curve
$E$ with equation $y^2=x^3+ax+b$.  
\item Try to compute $kP$.  If you find a divisor, cheer! Otherwise, try again until you've tried $h$ times, then give up.
\end{enumerate}
Here $v$ serves as bound on the smallest prime factor of $N$ and $w$ a bound on the largest prime divisor of the group of points mod that prime. 
\end{frame}
\begin{frame}{Analysis of ECM}
The method described above will find a divisor provided that the following conditions hold:
\begin{enumerate}
\item $N$ has a prime divisor $p\le v$.
\item The elliptic curve chosen is non-singular mod $p$.
\item The order of the group of points on that curve is $w$-smooth.
\item For some \textit{other} divisor $q$ of $N$, $E$ is non-singular mod $q$ and the order of $P$ mod $q$ is NOT divisible by the
largest prime divisor of the order of $P$ mod $p$. 
\end{enumerate}

How likely is this to happen?  There is a trade off between time spent on one curve $v,w$ and number of times you try $h$.

\end{frame}
\begin{frame}{Analysis of ECM, cont'd}
Lenstra uses  properties of elliptic curves to show the following. Let
$$
u=|\{ s: |s-p-1|<\sqrt{p} \hbox{\rm\ and $s$ is $w$-smooth}\}| 
$$
So $u$ is the number of $w$-smooth numbers that could possibly be the order of an elliptic curve mod $p$.
Then the chance that a single attempt at Lenstra's ECM method works (i.e. that a particular $(a,x,y)$ succeeds) is closely related to
the chance of a smooth group order, or $u/(2\sqrt{p}+1)$. 

\bigskip\noindent
Conjecture: The chance that a randomly chosen integer in the range $(x-\sqrt{x}+1,x+\sqrt{x}+1)$ is $w$-smooth is the same
as the chance that a randomly chosen integer of that size is $w$-smooth.
\end{frame}
\begin{frame}{Analysis of ECM, cont'd}
\begin{theorem} Suppose $N$ has two prime divisors, that $p$ is the smallest prime divisor of $N$, and that $N$ is prime to $6$.
Let $f(w)$ be the probability that a randomly chosen integer in the range $$(p+1-\sqrt{p},p+1+\sqrt{p})$$
is $w$-smooth.  Assuming $v\ge p$,
then there is an explicitly computable constant $c$ so that the chance of success of the ECM with parameters $v,w,h$
is 
$$1-c^{-hf(w)/\log(v)}.$$
\end{theorem}
\end{frame}
\begin{frame}{Analysis of ECM, cont'd}
\bigskip\noindent
Meanwhile, the running time for the algorithm is 
$$O(hw\log(v)M(N))$$
 because you have to compute $kP$,
and 
$$
\log k\le w\log(v).
$$
\end{frame}
\begin{frame}{Analysis of ECM, cont'd}
Let $$
L(x)=e^{\sqrt{\log(x)\log\log(x)}}.
$$

\bigskip\noindent
Lenstra invokes a famous result of Canfield, Erd\"{o}s, and Pomerance that states that the probability that a random positive integer
$s\le x$  is $L(x)^{\alpha}$-smooth is (asymptotically) 
$$
L(x)^{\frac{1}{2\alpha}+o(1)}
$$
\end{frame}
\begin{frame}{Analysis of ECM, cont'd}
One way to think of the problem is: for a fixed chance of success, minimize running time.  In other words,
for a fixed value of $hf(w)$, minimize $hw$.  This boils down to minimizing $w/f(w)$.

\bigskip\noindent
The CEP result says that
$$
w/f(w)= O(L(p)^{\frac{1}{2\alpha}+\alpha+o(1)}
$$
so the minimum value happens when $\alpha=\frac{1}{\sqrt{2}}$, and, in that case.
$$
w/f(w)=O(L(p)^{\sqrt{2}}).
$$

\bigskip\noindent
In other words, for a fixed chance of success, the running time behaves like
$$
L(p)^{\sqrt{2}}=e^{\sqrt{((2+o(1))\log p\log\log p}}
$$
where $p$ is the smallest prime factor of $N$.
\end{frame}
\begin{frame}{Further Discussion}
To make ECM work in practice there are many refinements, including:
\begin{enumerate}
\item Using highly optimized elliptic curve operations and modular arithmetic;
\item Careful choice of parameters (smoothness bound, for example).
\end{enumerate}

\href{http://swc.math.arizona.edu/aws/2006/notes.html}{Bernstein's notes} from the 2006 Winter School address some of these optimizations.
\end{frame}
\end{document}